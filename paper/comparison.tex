\section{Comparison with Existing Systems}

In our survey of the literature, we found no such research for the explicit task of searching for conversations. More so, we did not identify any such research-oriented system for searching chat conversations, or messages for that matter. However, in terms of the the searching for conversations research tasks established in Section~\ref{sec:importance}, we have identified a number of works that have addressed these tasks, and would have benefited from the PECAN ecosystem.

For the task of conversation aggregations ...


For the task of conversation scoring ...

For the task of query suggestion ...

For the task of conversation summarisation ...

For the task of related conversations ...


\cite{shen2006thread}, the authors used a single-pass clustering algorithm and it's variations to do thread detection.
	 
In \cite{khan2002mining}, the authors also mentioned three ways to extract items or topics from a chat messages. The simplest approach is to separate the chat based on time range. This approach does not work very well because conversations are usually not continuous. The second approach is to use topic segmentors to identify conversational flows as threads. However, much information can be lost when separating contents into groups. The revised approach is to classify threads into thread starts and non-thread starts and attach non-thread starts to appropriate thread starts. 
	
The article \cite{magnani2012conversation} introduces a model to define on-line conversation and distinguish it from off-line conversation.
	
	